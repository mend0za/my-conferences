\section{XMPP против других IM}

\subsection{Подъём и падение XMPP}
\begin{frame}{Подъём \ldots и падение XMPP}
  \begin{block}{Подъём}
    Технология XMPP заняла свои позиции. \\
    Сервисы: Facebook, GTalk, VKontakte, WhatsApp. \\
    Бизнес-решения: MS Lync, Cisco Unified Presence.

    Позволяют подключится по XMPP или основаны на нём
  \end{block} \pause

  \begin{block}{\ldots и падение}
    Разработчики корпоративных и открытых решений множат мало функциональные и/или несовместимые решения.
  \end{block}
\end{frame}

\begin{frame}{Метнём коричневые комки правды}
  \begin{block}{Серверные красавчики}
    \begin{itemize}
      \item Отключен s2s : Facebook, Vkontakte, Odnoklassniki
      \item Нельзя поменять VCard : почти все
      \item Странное поведение серверов: Google Talk
      \item Нет server-side history
      \item Нет поддержки групп : см s2s
    \end{itemize} 
  \end{block}\pause

  \begin{block}{Клиентские красавчики}
    \begin{itemize}
      \item нет поддержки Service Discovery (imo, im+, xabber)
      \item нет поддержки multi user chat
      \item нет групп в ростере (imo)
    \end{itemize}
    Особенно мало умеют многопротокольные клиенты\footnote{где XMPP лишь одна из опций}.
  \end{block}
\end{frame}

\begin{frame}{Итого}
  \begin{block}{Теоретическое богатство возможностей XMPP}
  \end{block}

  \pause

  \begin{block}{\ldots недоступно в качественном коробочном продукте.}
  \end{block}

  \pause

  \begin{block}{Часто требует ощутимых усилий от оператора сервера \ldots}
  \end{block}
  
  \pause

  \begin{block}{или даже от конечного пользователя \ldots}
  \end{block}
\end{frame}

\subsection{Заметки о войнах IM}

\begin{frame}{1я Мировая Война IM}
    \alert{Время}: от конца 90-х до (приблизительно) 2005. 

    \alert{Предпосылки}: Широкое распространение Internet. Dial-Up. 
    
    \alert{Начало}: Появился ICQ. 
      
    \alert{Ключевой функционал}: передача текста, сообщения и групповые чаты

    \alert{Кто?}: ICQ, AIM, MSN, Yahoo, IRC

    \alert{Конец войны}: интеграция протоколов и клиентов
\end{frame}

\begin{frame}{Подходы к IM Protocol Hell}
  2 магистральных подхода, cформировались в ходе 1й Войны:

  \begin{block}{Многопротокольные клиенты}
    Miranda, Pidgin, Trillian, Empathy/Telepathy
  \end{block}

  \pause

  \begin{block}{Сервер-интегратор}
    Клиент работает по 1 протоколу. 
    
    Сервер транслирует протоколы в 1.

    Известные реализации: XMPP/Jabber, MS Lync, imo.im, Cisco Unified Meeting Place
  \end{block}

\end{frame}

\begin{frame}{Jabber/XMPP - way}
  \begin{itemize}
    \item Реверс-инжениринг закрытых протоколов
    \item Написаны транспорты ко всем известным протоколам
  \end{itemize}
\end{frame}

\begin{frame}{2я Мировая Война IM}
    \alert{Время}:  2005. 

    \alert{Предпосылки}: Широкополосный скоростной интернет
    
    \alert{Начало}: Появился Skype. 
      
    \alert{Ключевой функционал}: голосовые и видео-звонки

    \alert{Кто?}: Skype, WhatsApp, Google Hangout, Viber, SIP

    \alert{Конец войны}: продолжается
\end{frame}

{
\usebackgroundtemplate{\includegraphics[width=\paperwidth,height=\paperheight]{goryaschii_tank_u_reki}}
\begin{frame}{Репортаж из горящего танка\footnote{\alert{Hint: внутри WoT Chat - Jabber}}}

  \alert{
    \begin{enumerate}
      \item Cтеки протоколов и технологий: SIP\footnote{\alert{Skype, Viber}} и XMPP\footnote{\alert{WhatsApp}}

      \item Полностью закрытые решения без внешнего API

      \item Производитель воспринимает своих пользователей как свою собственность 

    \end{enumerate}
  }

\end{frame}
}

{
\usebackgroundtemplate{\includegraphics[width=\paperwidth,height=\paperheight]{facebook-logo}}
\begin{frame}{Анальное огораживание, Facebook-way}
  \begin{block}{Есть}
    \begin{enumerate}
      \item чат с пользователями в Facebook
      \item публикация статуса онлайн/оффлайн
      \item несколько соединений одновременно
    \end{enumerate}
  \end{block} 
  \pause

  \begin{block}{Нет / Не работает}
    \begin{enumerate}
      \item S2S
      \item сервисов и транспортов
      \item групп и управления группами
      \item видео и аудио
      \item не работает VCard 
      \item запутанная схема подключения
      \item нельзя публиковать и читать материалы в ленте
      \item нельзя подключать новых пользователей
    \end{enumerate}
  \end{block}
\end{frame}
}

{
\usebackgroundtemplate{\includegraphics[width=\paperwidth,height=\paperheight]{google-talk-logo}}
\begin{frame}{Былинный отказ: Google Talk}
  \alert{Whats going on? Факты.}

  \begin{itemize}
    \item с 2006 до 2013 - GTalk стал крупнейшим сервером XMPP/Jabber
    \item \alert{ТОЛЬКО} за счёт навязывания Google Talk в нагрузку к GMail и Android\footnote{см словарь - ``Связанная покупка''}. \pause
    \item Голос, видео и звонки на телефон - позже чем в скайп
    \item реализация - слабо совместима с остальным XMPP \pause
    \item Google не замечен в работе над стандартами XMPP
    \item Единственный вклад - Jingle и GSoC
  \end{itemize}
  \pause
  \alert{Вывод: компания сильно не вкладывалась в XMPP, и выбросила его без сожаления. Вслед за RSS Reader}

\end{frame}
}

